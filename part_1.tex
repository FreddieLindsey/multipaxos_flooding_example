%------------------------------------------------------------------------------
%  DOCUMENT CONFIGURATION
%------------------------------------------------------------------------------

\documentclass[a4paper]{article}
\usepackage[landscape, margin = 1in]{geometry}

%------------------------------------------------------------------------------
%  PACKAGES
%------------------------------------------------------------------------------

\usepackage{graphicx} % Required to insert images
\usepackage{rotating} % Required for sideways figure
\usepackage{parskip} % Required for paragraph styling
\usepackage{float} % For having figures inline
\usepackage{amsmath}
\usepackage{amssymb}
\usepackage{hyperref} % For links
\usepackage{blindtext}
\usepackage{outlines}
\usepackage{tabularx}

\usepackage{tikz}
\usepackage{environ}
\restylefloat{figure}

\makeatletter
\newsavebox{\measure@tikzpicture}
\NewEnviron{scaletikzpicturetowidth}[1]{%
  \def\tikz@width{#1}%
  \def\tikzscale{1}\begin{lrbox}{\measure@tikzpicture}%
  \BODY
  \end{lrbox}%
  \pgfmathparse{#1/\wd\measure@tikzpicture}%
  \edef\tikzscale{\pgfmathresult}%
  \BODY
}
\makeatother

\begin{document}

% Definition of blocks:
\tikzset{%
  block/.style    = {draw, thick, rectangle, minimum height = 3em,
    minimum width = 3em},
  sum/.style      = {draw, circle, node distance = 2cm}, % Adder
  input/.style    = {coordinate}, % Input
  output/.style   = {coordinate} % Output
}

\pagestyle{empty}

%------------------------------------------------------------------------------

Frederick Lindsey (fl1414) and Cyrus Vahidi (cv114)

\begin{figure}[H]
  \centering
  \begin{scaletikzpicturetowidth}{\textwidth}
  \begin{tikzpicture}[scale = \tikzscale]

  % Whole system overview

  \node[above] at (0, 10) {System};
  \draw (-20, -10) rectangle (20, 10);

  % Client processes

  \node[align=center] at (-18, 7) {Client};
  \draw (-19, 6) rectangle (-17, 8);

  \node[align=center] at (-18, 0) {Client};
  \draw (-19, -1) rectangle (-17, 1);

  \node[align=center] at (-18, -7) {Client};
  \draw (-19, -8) rectangle (-17, -6);

  % Server nodes

  \node[above right] at (-13, 9) {Server};
  \draw (-13, 5) rectangle (19, 9);

  \node[above right] at (-13, 2) {Server};
  \draw (-13, -2) rectangle (19, 2);

  \node[above right] at (-13, -5) {Server};
  \draw (-13, -9) rectangle (19, -5);

  % Replica processes

  \node[align=center] at (-11, 0) {Replica};

  \end{tikzpicture}
  \end{scaletikzpicturetowidth}
  \caption{
    System structure representation using client, database, replica, server, and system modules. System shows use with 3 clients, and 3 servers, and is expanded as such.
  }
\end{figure}

%------------------------------------------------------------------------------

\end{document}
